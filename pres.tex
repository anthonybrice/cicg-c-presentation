\documentclass[pdf]{beamer}

\usepackage{newfloat}

\usepackage{graphicx}
\usepackage{amsmath, amsthm, amssymb, amsfonts}
\usepackage{multicol}
\usepackage{multirow}
%\usepackage{fancyvrb}
%\fvset{fontsize=\normalsize}
\usepackage[inline]{enumitem}

\usepackage[newfloat]{minted}
\setminted{bgcolor=lightgray,fontsize=\normalsize,frame=leftline,linenos}
% \newmintinline[cinline]{c}{fontsize=\normalsize}
% \newmintinline[shinline]{sh}{fontsize=\normalsize,fgcolor=black}
\setmintedinline{bgcolor=lightgray,fontsize=\normalsize}

\usepackage[T1]{fontenc}
\usepackage[boldsans]{ccfonts}
\usepackage[euler-digits,euler-hat-accent]{eulervm}
\usepackage{cabin}
\usepackage[ activate={true,nocompatibility},final,tracking=true,%
  kerning=true,spacing=]{microtype}
\usepackage{inconsolata}
\setbeamertemplate{navigation symbols}{}

\renewcommand{\theFancyVerbLine}{\ttfamily {\footnotesize \arabic{FancyVerbLine}}}

\usepackage{enumitem}
\setlist[itemize,1]{label={\fontfamily{cmr}\fontencoding{T1}\selectfont\textbullet}}
\setlist[itemize,2]{label={\fontfamily{cmr}\fontencoding{T1}\selectfont\textopenbullet}}

\mode<presentation>{}

\title{C Bootcamp}
\author{CI Computer Girls}

\begin{document}
\begin{frame}
  \titlepage%
\end{frame}

\begin{frame}[fragile]
  \frametitle{Hello World}

\begin{minted}{c}
#include <stdio.h>

main() {
  printf("Hello, world!\n");
}
\end{minted}
\\
  \begin{itemize}
  \item A C program consists of \textit{functions} and \textit{variables}.
    \pause
  \item A function contains \textit{statements} that specify the
    computing operations to be done.
    \pause
  \item Variables store values to be used during computation.
    \pause
  \item Normally you can name functions whatever you like, but every program
    must contain a function named \mintinline{text}{main}.
  \end{itemize}
\end{frame}

\begin{frame}[fragile]
  \frametitle{Hello World}

\begin{minted}{c}
#include <stdio.h>

main() {
  printf("Hello, world!\n");
}
\end{minted}
\\
  \begin{itemize}
  \item In this example \mintinline{c}{printf} is a function that takes a
    \textit{character string} as its argument.
    \pause
  \item Copy the code above into an empty file \mintinline{text}{hello.c} in your
    \mintinline{text}{task1} directory (We'll help you find it.), and then from your
    terminal:
    \pause
  \end{itemize}
\bigskip
\begin{minted}[linenos=false]{text}
# cd ~/Desktop/bootcamp/task1
# gcc hello.c
# ./a.out
\end{minted}

\end{frame}

\begin{frame}[fragile]
  \frametitle{Prompts}

  From your terminal,
\begin{minted}[linenos=false]{text}
# cd ../task2
\end{minted}
\\
then open the file \mintinline{text}{prompt.c} in your text editor. You should
see the following:

\pause

\bigskip

\begin{minted}[fontsize=\footnotesize]{c}
#include <stdio.h>

main() {
  char name[40];
  printf("Enter your name:\n");

  // YOUR TASK: Prompt the user for their name say hello.

}
\end{minted}

\end{frame}

\begin{frame}[fragile]
  \frametitle{Prompts}

\begin{minted}[fontsize=\footnotesize]{c}
#include <stdio.h>

main() {
  char name[40];
  printf("Enter your name:\n");

  // YOUR TASK: Prompt the user for their name say hello.

}
\end{minted}

  \begin{itemize}
  \item For this task, we'll make use of a new function
\begin{minted}[linenos=false,bgcolor=lightgray,frame=none]{c}
scanf(char* format, ...)
\end{minted}
    \pause
  \item \mintinline{c}{scanf} reads characters from your terminal, interprets them
    according to the \mintinline{text}{format} you provide (consult your cheatsheet), and stores the
    results in the remaining arguments.
    \pause
  \item For example, to store a user-given string in \mintinline{text}{name},
\begin{minted}[linenos=false,frame=none]{c}
scanf("%s", name);
\end{minted}
  \end{itemize}

\end{frame}

\begin{frame}[fragile]
  \frametitle{Prompts}
\begin{minted}[fontsize=\footnotesize]{c}
#include <stdio.h>

main() {
  char name[40];
  printf("Enter your name:\n");

  // YOUR TASK: Prompt the user for their name say hello.

}
\end{minted}
  \begin{itemize}
  \item For example, to store a user-given string in \mintinline{text}{name},
\begin{minted}[linenos=false,frame=none]{c}
scanf("%s", name);
\end{minted}
  \item Similarly, \mintinline{c}{printf} can be given format specifiers in its
    first argument and will print the rest of its arguments accordingly.
\begin{minted}[linenos=false,frame=none]{c}
printf("Goodbye %s", name);
\end{minted}
    \pause
  \item Complete your task (Ask for your help if you're stuck!), and run your program.
  \end{itemize}
\end{frame}

\begin{frame}[fragile]
  \frametitle{Arguments}
  From your terminal,
\begin{minted}[linenos=false]{text}
# cd ../task3
\end{minted}
\\
  then open the file \mintinline{text}{arguments.c} in your text editor. You
  should see the following:
  \pause
\begin{minted}[fontsize=\footnotesize]{c}
#include <stdio.h>
#include <stdlib.h>

int main(int argc, char* argv[]) {
  if (argc < 3) {
    printf("Usage: %s <name> <integer>\n", argv[0]);
    return -1;
  }

  // YOUR TASK: Read the user's name and an integer
  // from command line arguments, then say hello
  // to the user as many times as given by the integer.

}
\end{minted}

\end{frame}

\begin{frame}[fragile]
  \frametitle{Arguments}

\begin{minted}{c}
int main(int argc, char* argv[]) {
  ...
}
\end{minted}

  \begin{itemize}
  \item Note that our \mintinline{c}{main} has grown a little.
    \pause
  \item The first \mintinline{c}{int} tells us that this function will return
    an integer.
    \pause
  \item \mintinline{c}{int argc} and \mintinline{c}{char* argv[]} are
    parameters to \mintinline{c}{main}.
    \pause
    \begin{itemize}
    \item \mintinline{c}{char* argv[]} is an array of strings containing all the
      arguments we'll pass when we run our program. (More on that later.)
      \pause
    \item \mintinline{c}{int argc} is an integer indicating the length of
      \mintinline{c}{argv} or the number of strings contained within.
    \end{itemize}
  \end{itemize}
\end{frame}

\begin{frame}[fragile]
  \frametitle{Arguments}
\begin{minted}{c}
int main(int argc, char* argv[]) {
  ...
}
\end{minted}

  \bigskip

  For example, if we invoke our program as follows:
\begin{minted}[linenos=false]{text}
# ./a.out CiComputerGirls 5
\end{minted}

  \pause

  \bigskip

  \begin{itemize}
  \item Then \mintinline{c}{argc} contains the integer 3.
    \pause
  \item \mintinline{c}{argv[0]} contains the string \mintinline{c}{"a.out"}.
    \pause
  \item \mintinline{c}{argv[1]} contains the string \mintinline{c}{"CiComputerGirls"}.
    \pause
  \item \mintinline{c}{argv[2]} contains the string \mintinline{c}{"5"}.
  \end{itemize}

\end{frame}

\begin{frame}[fragile]
  \frametitle{Arguments}

\begin{minted}[fontsize=\footnotesize]{c}
int main(int argc, char* argv[]) {
  if (argc < 3) {
    printf("Usage: %s <name> <integer>\n", argv[0]);
    exit(-1);
  }
  ...
}
\end{minted}

  \begin{itemize}
  \item In the given code, we examine \mintinline{c}{argc} in the condition of
    our if-statement to ensure our program was passed the correct number of
    arguments.
    \pause
  \item And if not, we print a helpful message and exit with an error code.
    \pause
  \item Note that our helpful message prints the value of
    \mintinline{c}{argv[0]}. The first string in \mintinline{c}{argv} will always
    be the name of your program.
  \end{itemize}

\end{frame}

\begin{frame}[fragile]
  \frametitle{Arguments}
\begin{minted}[fontsize=\footnotesize]{c}
int main(int argc, char* argv[]) {
  ...

  // YOUR TASK: Read the user's name and an integer
  // from command line arguments, then say hello
  // to the user as many times as given by the integer.

}

\end{minted}

\bigskip

To complete your task,

  \begin{itemize}
  \item Use the function \mintinline{c}{atoi} to convert the value of
    \mintinline{c}{argv[2]} into an \mintinline{c}{int}.
\begin{minted}[frame=none,linenos=false]{c}
int atoi(char* s)
\end{minted}
    \pause
  \item \mintinline{c}{atoi} converts the string \mintinline{c}{s} into an
    \mintinline{c}{int}. For example,
\begin{minted}[frame=none,linenos=false]{c}
int five = atoi("5");
\end{minted}
    \pause
  \item Then use a for- or while-loop to print your message as many times as needed.
  \end{itemize}

\end{frame}

\begin{frame}{Pointers}

\end{frame}

\begin{frame}{Arrays}

\end{frame}

\begin{frame}{Structs}

\end{frame}

\begin{frame}{Headers}

\end{frame}
\end{document}
