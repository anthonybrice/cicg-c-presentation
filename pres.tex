\documentclass[pdf]{beamer}

\usepackage{newfloat}

\usepackage{graphicx}
\usepackage{amsmath, amsthm, amssymb, amsfonts}
\usepackage{multicol}
\usepackage{multirow}
%\usepackage{fancyvrb}
%\fvset{fontsize=\normalsize}
\usepackage[inline]{enumitem}

\usepackage[newfloat]{minted}
\setminted{bgcolor=lightgray,fontsize=\footnotesize,frame=leftline,linenos}
% \newmintinline[cinline]{c}{fontsize=\normalsize}
% \newmintinline[shinline]{sh}{fontsize=\normalsize,fgcolor=black}
\setmintedinline{bgcolor=lightgray,fontsize=\normalsize}

\usepackage[T1]{fontenc}
\usepackage[boldsans]{ccfonts}
\usepackage[euler-digits,euler-hat-accent]{eulervm}
\usepackage{cabin}
\usepackage[ activate={true,nocompatibility},final,tracking=true,%
  kerning=true,spacing=]{microtype}
\usepackage{inconsolata}
\setbeamertemplate{navigation symbols}{}

\renewcommand{\theFancyVerbLine}{\ttfamily {\footnotesize \arabic{FancyVerbLine}}}

\usepackage{enumitem}
\setlist[itemize,1]{label={\fontfamily{cmr}\fontencoding{T1}\selectfont\textbullet}}
\setlist[itemize,2]{label={\fontfamily{cmr}\fontencoding{T1}\selectfont\textopenbullet}}

\mode<presentation>{}

\title{C Bootcamp}
\author{CI Computer Girls}

\begin{document}
\begin{frame}
  \titlepage%
\end{frame}

\begin{frame}[fragile]
  \frametitle{Hello World}

\begin{minted}{c}
#include <stdio.h>

main() {
  printf("Hello, world!\n");
}
\end{minted}
\\
  \begin{itemize}
  \item A C program consists of \textit{functions} and \textit{variables}.
    \pause
  \item A function contains \textit{statements} that specify the
    computing operations to be done.
    \pause
  \item Variables store values to be used during computation.
    \pause
  \item Normally you can name functions whatever you like, but every program
    must contain a function named \mintinline{text}{main}.
  \end{itemize}
\end{frame}

\begin{frame}[fragile]
  \frametitle{Hello World}

\begin{minted}{c}
#include <stdio.h>

main() {
  printf("Hello, world!\n");
}
\end{minted}
\\
  \begin{itemize}
  \item Copy the code above into an empty file \mintinline{text}{hello.c} in your
    \mintinline{text}{task1} directory (We'll help you find it.), and then from your
    terminal:
    \pause
  \end{itemize}
\bigskip
\begin{minted}[linenos=false]{text}
# cd ~/Desktop/bootcamp/task1
# gcc hello.c
# ./a.out
\end{minted}

\end{frame}

\begin{frame}
\frametitle{Prompts}

\end{frame}

\begin{frame}[fragile]{Arguments}

\begin{minted}{c}
#include <stdio.h>

int main(int argc, char** argv)
{
  // argc is always at least 1, because
  // the first argument value is the program name
  if (argc < 2) {
    fprintf(stderr, "Usage: %s <name>\n", argv[0]);
    return -1;
  }

  fprintf(stdout, "Hello, %s!\n", argv[1]);
  return 0;
}
\end{minted}

\end{frame}

\begin{frame}{Pointers}

\end{frame}

\begin{frame}{Arrays}

\end{frame}

\begin{frame}{Structs}

\end{frame}

\begin{frame}{Headers}

\end{frame}
\end{document}
