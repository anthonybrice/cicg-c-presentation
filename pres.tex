\documentclass[pdf]{beamer}

\usepackage{newfloat}

\usepackage{graphicx}
\usepackage{amsmath, amsthm, amssymb, amsfonts}
\usepackage{multicol}
\usepackage{multirow}
%\usepackage{fancyvrb}
%\fvset{fontsize=\normalsize}
\usepackage[inline]{enumitem}

\usepackage[newfloat]{minted}
\setminted{bgcolor=lightgray,fontsize=\footnotesize,frame=leftline,linenos}

\usepackage[T1]{fontenc}
\usepackage[boldsans]{ccfonts}
\usepackage[euler-digits,euler-hat-accent]{eulervm}
\usepackage{cabin}
\usepackage[ activate={true,nocompatibility},final,tracking=true,%
  kerning=true,spacing=]{microtype}
\usepackage{inconsolata}
\setbeamertemplate{navigation symbols}{}

\renewcommand{\theFancyVerbLine}{\ttfamily {\scriptsize \arabic{FancyVerbLine}}}

\mode<presentation>{}

\title{C Bootcamp}
\author{CI Computer Girls}

\begin{document}
\begin{frame}
  \titlepage%
\end{frame}

\begin{frame}[fragile]{Hello World}

\begin{minted}{c}
// Includes the code from the standard (std for short)
// library needed to do input and output (io).
#include <stdio.h>

// The main function.
// In C, the main function returns an error code, which
// is an int. 0 means 'no error'.
int main() {
  // printf prints its argument to the terminal.
  printf("Hello, world!\n");

  return 0;
}
\end{minted}

\end{frame}

\begin{frame}[fragile]{Arguments}

\begin{minted}{c}
#include <stdio.h>

int main(int argc, char** argv)
{
  // argc is always at least 1, because
  // the first argument value is the program name
  if (argc < 2) {
    fprintf(stderr, "Usage: %s <name>\n", argv[0]);
    return -1;
  }

  fprintf(stdout, "Hello, %s!\n", argv[1]);
  return 0;
}
\end{minted}

\end{frame}

\begin{frame}{Pointers}

\end{frame}

\begin{frame}{Arrays}

\end{frame}

\begin{frame}{Structs}

\end{frame}

\begin{frame}{Headers}

\end{frame}
\end{document}
